\documentclass[aspectratio=169,bigger,xcolor={table}]{beamer}
\usepackage[magyar]{babel}
\usepackage{t1enc}
\usepackage{hulipsum}
\usepackage{xcolor}
\usepackage{amsthm}

\author{Harsányi Balázs}
\title{Gyakorlat}
\subtitle{több szó}

\newtheorem{tet}{Tétel}

\AtBeginSection{\frame{\sectionpage \tableofcontents}}
\usecolortheme{seahorse}
\usetheme{Boadilla}
\setbeamercovered{transparent}

\usepackage{hyperref}
\begin{document}

\begin{frame}[plain]
\maketitle
\end{frame}

\section{első fel}

\subsection{elso egy}

\begin{frame}[allowframebreaks]{Írjunk bele}{alcím}
pár szót\\
\hulipsum[1-3]
\only<2>{hulipsum[4]}
\transboxin<1>
\end{frame}

\subsection{elso kett}

\begin{frame}{Második frame}{alalcím}
2. szavak
\end{frame}

\begin{frame}[fragile]{3. frame}{legalsócím}
harmadik frame szavai \\
\verb|\begin{frame}|
\end{frame}

\section{2. fel}


\begin{frame}{CÍME}
\begin{columns}

\begin{column}{5cm}
\begin{itemize}
\item<3-> első 
\item<4> második
\item harmadik 
\end{itemize}
\begin{enumerate}
\item<1-2> szám
\item number
\item valami
\end{enumerate}
\transduration{2}
\end{column}

\begin{column}{5cm}
\begin{figure}
\caption{ez egy kép}
\includegraphics[scale=0.1]{kutyakep}
\end{figure}
\end{column}

\end{columns}
\end{frame}

\begin{frame}
\begin{block}{block cím}
block szövege
\end{block} \pause
\begin{alertblock}{alertblock cím}
példa szavak
\end{alertblock} \pause \pause
\begin{exampleblock}{examplebock cím}
valami írás
\end{exampleblock}
\begin{block}{}
címtelen szöveg
\end{block}
\end{frame}

\begin{frame}
\begin{tet}
Theorem környezet
\end{tet} \pause
\begin{proof}[Pitagorasz tétel bizonyítása]
sok háromszög vagy mi
\end{proof}
\end{frame}

\subsection{semiverbatim}

\begin{frame}
\begin{semiverbatim}

\\begin\{itemize\}

\\item szó1

\\item szó2

\\begin\{itemize\}

\\item szó 3

\\item szó4

\\end\{itemize\}

\\end\{itemize\}

\end{semiverbatim}
\end{frame}

\begin{frame}
\begin{figure}
\caption{ez egy kép}
\only<1>{\includegraphics[scale=0.1]{kutyakep}}
\only<2>{\includegraphics[scale=0.5]{kutyakep2}}
\end{figure}
\onslide<2>{szöveg 1 }
\only<2>{valami 2}
kitöltés3
\visible<2>{legyen itt írás}
\visible<1>{kellet még példa}
több írott szöveg
\end{frame}


\end{document}